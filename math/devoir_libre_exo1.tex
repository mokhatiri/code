\documentclass{article}
\title{Exercice 1: devoir libre}
\author{Mohammed Khatiri}
\begin{document}
\maketitle
\newpage 

\section{Exercice 1:}

Soit E un K-espace vectoriel de dimension finie n et soit B une base de E. Soient V =
(v1,...,vp) une famille de vecteurs de E et v un vecteur de E. Soient T une forme échelon
obtenue par la méthode de Gauss appliquée à $Row^V_B$ et $T_v$ la matrice à p + 1 lignes, de p
premières lignes, les lignes de T et de dernière ligne, la ligne des coordonnées de v dans
B.

\subsection{Question 1:}

Montrer que la famille formée par les vecteurs de E dont les lignes des coordonnées
correspondent aux lignes non nulles de T forme une base de Vect(V).
\\
\section*{Réponse:}\
prenons v un vecteur de la famille de $Vect(V)$, il s'écrit alors sous la forme : $\sum_{i=1}^{n}\alpha_{i}v_{i}$.\newline
chaque vecteur des lignes non nulles de T, s'écrivent sous la forme: $u_i = \sum_{j=i}^{p}\beta_{i}v_i$ ;puisque chaque ligne de la matrice en forme échelonée est une combinaison linéaire des lignes initialle précédente de la matrice.\newline
on peut montrer alors que les vecteurs $v_i$ sont des combinaison linéaire des vecteur $u_i$:\newline

\end{document}